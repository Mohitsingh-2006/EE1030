
%iffalse
\let\negmedspace\undefined
\let\negthickspace\undefined
\documentclass[journal,12pt,twocolumn]{IEEEtran}
\usepackage{cite}
\usepackage{amsmath,amssymb,amsfonts,amsthm}
\usepackage{enumitem,multicol}
\usepackage{algorithmic}
\usepackage{graphicx}
\usepackage{textcomp}
\usepackage{xcolor}
\usepackage{txfonts}
\usepackage{listings}
\usepackage{enumitem}
\usepackage{mathtools}
\usepackage{gensymb}
\usepackage{comment}
\usepackage[breaklinks=true]{hyperref}
\usepackage{tkz-euclide} 
\usepackage{listings}
\usepackage{gvv}                                        
%\def\inputGnumericTable{}                                 
\usepackage[latin1]{inputenc}                                
\usepackage{color}                                            
\usepackage{array}                                            
\usepackage{longtable}                                       
\usepackage{calc}                                             
\usepackage{multirow}                                         
\usepackage{hhline}                                           
\usepackage{ifthen}                                           
\usepackage{lscape}
\usepackage{tabularx}
\usepackage{array}
\usepackage{float}


\newtheorem{theorem}{Theorem}[section]
\newtheorem{problem}{Problem}
\newtheorem{proposition}{Proposition}[section]
\newtheorem{lemma}{Lemma}[section]
\newtheorem{corollary}[theorem]{Corollary}
\newtheorem{example}{Example}[section]
\newtheorem{definition}[problem]{Definition}
\newcommand{\BEQA}{\begin{eqnarray}}
\newcommand{\EEQA}{\end{eqnarray}}
\newcommand{\define}{\stackrel{\triangle}{=}}
\theoremstyle{remark}

% Marks the beginning of the document
\begin{document}
\bibliographystyle{IEEEtran}
\vspace{3cm}

\title{CH-10\\ Function}
\author{EE24BTECH11041-Mohit}
\maketitle
\newpage
\bigskip

\renewcommand{\thefigure}{\theenumi}
\renewcommand{\thetable}{\theenumi}

\section {C: MCQs with One correct Answer}
\begin{enumerate}
\item Suppose $f(x)=f(x+1)^2 $ for x $\geq$ -1.If $g(x)$ is the   function whose graph is the reflection of the graph $f(x)$ with respect to the line $y=x$,then $g(x)$ equal
\hfill(2002S)\\
\begin{multicols}{2}
\begin{enumerate}
    \item $-\sqrt{x}-1$,x$\geq$0
    \item $\frac{1}{(x+1)^2},x>-1$
    \item $\sqrt{x+1}$,$x\geq-1$
    \item $\sqrt{x} -1,x\geq 0$
\end{enumerate}
\end{multicols}
\item Let function $f:R\rightarrow R$ be defined by $f(x)=2x + \sin x$ for $x\in R$ ,then $f$ is
\hfill(2003S)\\
\begin{enumerate}
    \item one-to-one and onto
    \item one-to-one but not onto
    \item onto but not not onto
    \item neither one-to-one nor onto
\end{enumerate}
\item If $f:[0,\infty) \rightarrow [0,\infty)$,and $f(x)=\frac{x}{1+x}$ then $f$ is
\hfill(2003S)\\
\begin{enumerate}
    \item one-one and onto
    \item one-one but not onto
    \item onto but not one-one
    \item neither one-one nor onto
\end{enumerate}
\item Domain of definition of the functions\\ $f(x)=\sqrt{\sin^{-1}(2x)+\frac{\pi}{6}}$ for real valued $x$,is
\hfill(2003S)\\
\begin{multicols}{2}
\begin{enumerate}
    \item$\left[-\frac{1}{4},\frac{1}{2}\right]$\\
    \item$\left[-\frac{1}{2},\frac{1}{2}\right]$\\
    \item$\left(-\frac{1}{2},\frac{1}{9}\right)$\\
    \item$\left[-\frac{1}{4},\frac{1}{4}\right]$
\end{enumerate}
\end{multicols}
\item Range of the function $f(x)=\frac{x^2+x+2}{x^2+x+1};x\in R$ is
\hfill(2003S)\\
\begin{multicols}{2}
\begin{enumerate}
    \item $(1,\infty)$
    \item $(1,\frac{11}{7}]$
    \item $(1,\frac{7}{3}]$
    \item $(1,\frac{7}{5}]$
\end{enumerate}
\end{multicols}
\item If $f(x)=x^2+2bx+2c^2$ and $g(x)=-x^2-2cx+b^2$ such that $\min f(x)> \max g(x)$,then the relations between $b$ and $c$,is
\hfill(2003S)\\
\begin{multicols}{2}
\begin{enumerate}
    \item no real value of $b\& c$
    \item $0<c<b\sqrt{2}$
    \item $|c|<|b|\sqrt{2}$
    \item $|c|>|b|\sqrt{2}$
\end{enumerate}
\end{multicols}
\item If the function $f(x)=\sin x+\cos x$,$g(x)=x^2-1$,then $g(f(x))$ is invertible in the domain 
\hfill(2004S)\\
\begin{multicols}{2}
\begin{enumerate}
    \item $\left[0,\frac{\pi}{2}\right]$\\
    \item $\left[-\frac{\pi}{4},\frac{\pi}{4}\right]$\\
    \item $\left[-\frac{\pi}{2},\frac{\pi}{2}\right]$\\
    \item $[0,\pi]$
\end{enumerate}
\end{multicols}
\item If the function $f(x)$ and $g(x)$ are defined on $R \rightarrow R $ such that
\hfill(2005S)
\\$f(x)=$
$ \begin{cases}
    0, x \in rational\\
    x, x \in irrational 
\end{cases} $ \\
    $g(x)=$
$\begin{cases}
    0,x \in rational\\
    x,x \in irratoinal
\end{cases}$
then $(f-g)(x)$ is 
\begin{enumerate}
    \item one-one \& onto 
    \item neither one-one nor onto
    \item one-one but not onto 
    \item onto but not one-one
\end{enumerate}
\item $X$ and $Y$ are two sets and $f:X\rightarrow Y$.If $ \{f(c)=y;c \subset X,y \subset Y\} $ and $ \{f^{-1}(d)=X;d \subset Y,x \subset X \} $,then the true statement is 
\hfill(2005S)\\
\begin{multicols}{2}
\begin{enumerate}
    \item $f(f^{-1}(b))=b$
    \item $f^{-1}(f(a))=a$
    \item $f(f^{-1}(b))=b,b \subset y$
    \item $f^{-1}(f(a))=a,a \subset x$
\end{enumerate}
\end{multicols}
\item If $F(x)=\left( f\left( \frac{x}{2}\right) \right)^2 + \left(g\left (\frac{x}{2}\right) \right)^2$ where $f"(x)=-f(x)$ and $ g(x)=f'(x)$ and given that $F(5)=5$,then $F(10)$ is equal to 
\hfill(2006,-3M,-1)
\begin{multicols}{2}
\begin{enumerate}
    \item 5
    \item 10
    \item 0
    \item 15
\end{enumerate}
\end{multicols}
\item Let $f(x)=\frac{x}{(1+x^n)^\frac{1}{n}}$ for $ n\geq2$ and \begin{align*}g(x)= \underbrace{(f\circ f\circ...\circ f)(x)}_{\text{ f occurs n times}}\end{align*}.Then $\int x^{n-2} g(x)dx$ equals.
\hfill(2007-3 marks)\\
\begin{multicols}{2}
\begin{enumerate}
    \item$ \frac{1}{n(n-1)} (1+nx^n)^{1-\frac{1}{n}} $
    \item$ \frac{1}{n-1} (1+nx^n)^{1-\frac{1}{n}} $
    \item$ \frac{1}{n+1} (1+nx^n)^{1+\frac{1}{n}} $
    \item$ \frac{1}{n+1} (1+nx^n)^{1+\frac{1}{n}} $
\end{enumerate}
\end{multicols}
\item Let $f,g$ and $h$ be real-valued functions defined on the interval $[0,1]$ by $f(x)= e^{x^2} + e^{-x^2}$,$g(x)=xe^{x^2}+ e^{-x^2}$ and $h(x)=x^2e^{-x^2}$ .If $a,b$ and $c$ denote,respectively,the absolute maximum of $f,g$ and $h$ on [0,1],then 
\hfill(2010)
\begin{multicols}{2}
\begin{enumerate}
    \item $a=b$ and $b\neq c$
    \item $a=c$ and $a\neq b$
    \item $a\neq b$ and $c \neq b$
    \item $a=b=c$
\end{enumerate}
\end{multicols}
\item Let$f(x)=x^2$ and $g(x)=\sin x$ for all $x\in R$ .Then the set of all $x$ satisfying $(f\circ g\circ g\circ f)(x)=(g\circ g\circ f)(x)$,where $(f\circ g)(x)=f(g(x))$,is
\hfill(2011)
\begin{enumerate}
    \item $ \pm \sqrt{n\pi},n\in \{0,1,2....\}$
    \item $ \pm \sqrt{n\pi},n\in \{1,2,....\}$
    \item $ \frac{\pi}{2}+2n\pi,n \in\{...-2,-1,0,1,2...\} $
    \item $ 2n\pi,n\in \{...-2,-1,0,1,2...\}$
\end{enumerate}
\item The function $ f:[0,3] \rightarrow [1,29] $, defined by $ f(x)=2x^3-15x^2+39x+1 $,is 
\hfill(2012)
\begin{enumerate}
    \item one-one and onto 
    \item onto but not one-one 
    \item one-one but not onto
    \item neither one-one nor onto
\end{enumerate}
\end{enumerate}
\section {D: MCQs with One or More than One Correct}
\begin{enumerate}
\item If $y=f(x)=\frac{x+2}{x-1}$ then
\hfill(2008S)
\begin{enumerate}
    \item $x=f(y)$
    \item $f(1)=3$
    \item $y$ increase with $x$ for $x<1$
    \item $f$ is a rational function on $x$
\end{enumerate}
\end{enumerate}
\end{document}
