\let\negmedspace\undefined
\let\negthickspace\undefined
\documentclass[journal]{IEEEtran}
\usepackage[a5paper, margin=10mm, onecolumn]{geometry}
%\usepackage{lmodern} % Ensure lmodern is loaded for pdflatex
\usepackage{tfrupee} % Include tfrupee package
\setlength{\headheight}{1cm} % Set the height of the header box
\setlength{\headsep}{0mm}     % Set the distance between the header box and the top of the text
\usepackage{gvv-book}
\usepackage{gvv}
\usepackage{cite}
\usepackage{amsmath,amssymb,amsfonts,amsthm}
\usepackage{algorithmic}
\usepackage{graphicx}
\usepackage{textcomp}
\usepackage{xcolor}
\usepackage{txfonts}
\usepackage{listings}
\usepackage{enumitem}
\usepackage{mathtools}
\usepackage{gensymb}
\usepackage{comment}
\usepackage[breaklinks=true]{hyperref}
\usepackage{tkz-euclide} 
\usepackage{listings}
% \usepackage{gvv}                                        
\def\inputGnumericTable{}                                 
\usepackage[latin1]{inputenc}                                
\usepackage{color}                                            
\usepackage{array}                                            
\usepackage{longtable}                                       
\usepackage{calc}                                             
\usepackage{multirow}                                         
\usepackage{hhline}                                           
\usepackage{ifthen}                                           
\usepackage{lscape}
\renewcommand{\thefigure}{\theenumi}
\renewcommand{\thetable}{\theenumi}
\setlength{\intextsep}{10pt} % Space between text and floats
\numberwithin{equation}{enumi}
\numberwithin{figure}{enumi}
\renewcommand{\thetable}{\theenumi}
\begin{document}
\bibliographystyle{IEEEtran}
\title{Question 1-1.4-9p}
\author{EE24BTECH11041 - Mohit}
% \maketitle
% \newpage
% \bigskip
{\let\newpage\relax\maketitle}
\begin{enumerate}
	\item Let $\vec{A}\brak{4,2}$,$\vec{B}\brak{6,5}$ and $\vec{C}\brak{1,4}$ be the vertices of $\Delta$$ABC$.Find the coordinates of points Q and R on medians $BE$ and $CF$ respectively such that $BQ:QE=2:1$ and $CR:RF=2:1$.
\end{enumerate}
Solution:-\\
$F$ is the mid point of $AB$\\
\begin{align*}
F=\frac{A+B}{2}=\frac{\begin{pmatrix}4\\2\\ \end{pmatrix} + \begin{pmatrix}6\\5\\ \end{pmatrix}}{2}
=\begin{pmatrix}5\\ \frac{7}{2}\\ \end{pmatrix}
\end{align*}
$E$ is the mid point of $AC$\\
\begin{align*}
E=\frac{A+C}{2}=\frac{\begin{pmatrix}4\\2\\ \end{pmatrix} + \begin{pmatrix}1\\4\\ \end{pmatrix}}{2}
=\begin{pmatrix}\frac{5}{2}\\3\\ \end{pmatrix}
\end{align*}
By section formula,
\begin{align*}
    R=\frac{B+KA}{1+K}
\end{align*}
It is given that $\frac{BQ}{QE}=\frac{2}{1}$\\
So,\\
\begin{align*}
    Q=\frac{B+2E}{1+2}=\frac{\begin{pmatrix}6\\5\\ \end{pmatrix} + 2\begin{pmatrix}\frac{5}{2}\\3\\ \end{pmatrix}}{3}=\begin{pmatrix}\frac{11}{3}\\ \frac{11}{3}\\ \end{pmatrix}
\end{align*}
It is given that $\frac{CR}{RF}=\frac{2}{1}$\\
So,\\
\begin{align*}
    R=\frac{C+2F}{1+2}=\frac{\begin{pmatrix}1\\4\\ \end{pmatrix} + 2\begin{pmatrix}5\\\frac{7}{2}\\ \end{pmatrix}}{3}=\begin{pmatrix}\frac{11}{3}\\ \frac{11}{3}\\ \end{pmatrix}
\end{align*}
Hence, Co-ordinates of $Q$ and $R$ are\\
\begin{align*}
    \vec{Q}\brak{\frac{11}{3},\frac{11}{3}} \text{and}\ \vec{R}\brak{\frac{11}{3},\frac{11}{3}}
\end{align*}
\end{document}

