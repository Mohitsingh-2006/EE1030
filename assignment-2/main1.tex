%iffalse
\let\negmedspace\undefined
\let\negthickspace\undefined
\documentclass[journal,12pt,twocolumn]{IEEEtran}
\usepackage{cite}
\usepackage{amsmath,amssymb,amsfonts,amsthm}
\usepackage{algorithmic}
\usepackage{graphicx}
\usepackage{textcomp}
\usepackage{xcolor}
\usepackage{txfonts}
\usepackage{listings}
\usepackage{enumitem}
\usepackage{mathtools}
\usepackage{gensymb}
\usepackage{comment}
\usepackage[breaklinks=true]{hyperref}
\usepackage{tkz-euclide} 
\usepackage{listings}
\usepackage{gvv}                                        
%\def\inputGnumericTable{}                                 
\usepackage[latin1]{inputenc}                                
\usepackage{color}                                            
\usepackage{array}                                            
\usepackage{longtable}                                       
\usepackage{calc}                                             
\usepackage{multirow}                                         
\usepackage{hhline}                                           
\usepackage{ifthen}                                           
\usepackage{lscape}
\usepackage{tabularx}
\usepackage{array}
\usepackage{float}


\newtheorem{theorem}{Theorem}[section]
\newtheorem{problem}{Problem}
\newtheorem{proposition}{Proposition}[section]
\newtheorem{lemma}{Lemma}[section]
\newtheorem{corollary}[theorem]{Corollary}
\newtheorem{example}{Example}[section]
\newtheorem{definition}[problem]{Definition}
\newcommand{\BEQA}{\begin{eqnarray}}
\newcommand{\EEQA}{\end{eqnarray}}
\newcommand{\define}{\stackrel{\triangle}{=}}
\theoremstyle{remark}
\newtheorem{rem}{Remark}

% Marks the beginning of the document
\begin{document}
\bibliographystyle{IEEEtran}
\vspace{3cm}

\title{Title of your Document}
\author{Roll Number - Name}
\maketitle
\newpage
\bigskip

\renewcommand{\thefigure}{\theenumi}
\renewcommand{\thetable}{\theenumi}

\section {A -Fill in the Blanks}
\begin{enumerate}
    \item The area enclosed within the curve $|x|+|y| =1$ is ........
    \hfill(1981-2 Marks)
    \item $y = 10^x $ is the reflection of $y=\log x$ in the  line whose equation is........
    \hfill(1982-2 Marks)
    \item The set of lines $ax+by+c=0$,where $3a+2b+4c=0$ concurrent at the point........
    \hfill(1982-2 Marks)
    \item Given the points $A(0,4)$ and $B(0,-4)$,the equation of the locus of the point $p(x,y)$,such that \\
    $|AP-BP|=6$ is .......
    \hfill(1983-1 Marks)
    \item If $a,b and c$ are in A.P, then the straight line $ax +by +c=0$ will always pass through a fixed point whose coordinate are ........
    \hfill(1984-2 Marks)
    \item The orthocentre of the triangle formed by the lines $x+y=1,2x +3y=6$ and $4x-y+4=0$ lies in the quadrant number........
    \hfill(1985-2 Marks)
    \item Let the algebric sum of the perpendicular distances from the points $(2,0),(0,2)$ and $(1,1)$ to a variable straight line be zero;then the line passes through a fixed point whose coordinates are......
    \hfill(1991-2 Marks)
    \item the vertices of a triangle are $A(-1,-7)$,$B(5,1)$ and $C(1,10)$. The equation of the bisector of the angle $\angle{ABC}$ is.........
    \hfill(1993-2 marks)
\end{enumerate}
\section {B-True/False}
\begin{enumerate}
    \item The straight line $5x+4y=0$ passes through the point of intersection of the straight lines $x+2y-10=0$ and $2x+y+5=0$.
    \hfill(1983-1 Marks)
    \item The lines $2x+3y+19=0$ and $9x+6y-17=0$ cut the coordinates axes in concylic points.
    \hfill(1988-1 Marks)
\end{enumerate}
\section {C-MCQs with One Correct Answer}
\begin{enumerate}
    \item The points $(-a,-b)$,$(0,0)$,$(a,b)$ and $(a^2,ab)$ are:
    \hfill(1979)
    \begin{enumerate}
        \item collinear
        \item Vertices of a parallelogram
        \item Vertices of a rectangle 
        \item None of these
    \end{enumerate}
    \item The points of the $(4,1)$ undergoes the following three transformations successively.
    \hfill(1980)
    \begin{enumerate}
        \item Reflection about the line $y=x.$
        \item Translation through a distances of x-axis.
        \item Rotation through an $\frac{\pi}{4}$ about the origin in the counter clockwise direction.
    \end{enumerate}
    Then the final position of the point is given by the coordinates.
    \begin{enumerate}
        \item $\left(\frac{1}{\sqrt{2}},\frac{7}{\sqrt{2}} \right)$
        \item $(-\sqrt{2},7\sqrt{2})$
        \item $\left(-\frac{1}{\sqrt{2}},\frac{7}{\sqrt{2}} \right)$
        \item $(\sqrt{2},7\sqrt{2})$
    \end{enumerate}
    \item The straight lines $x+y=0$,$3x+y-4=0$,$x+3y-4=0$ from a triangle which is 
    \hfill(1983-1 Marks)
    \begin{enumerate}
        \item isosceles
        \item equilateral 
        \item right angled
        \item none of these
    \end{enumerate}
    \item If $p=(1,0)$,$Q=(-1,0)$ and $R=(2,0)$ are three given points,then locus of the points $S$ satisfying the relation\\
    $SQ^2+SR^2=2SP^2$ is 
    \hfill(1988-2 Marks)
    \begin{enumerate}
        \item a straight lines parallel to x-axis
        \item a circle passing through the origin
        \item a circle with the centre at the origin 
        \item a straight line parallel to y-axis
    \end{enumerate}
    \item Line $L$ has intercepts a and b on the coordinate axes.When the axes are rotated through a given angle,keeping the origin fixed,the same line $L$ has intercept $p$ and $q$,then 
    \hfill(1990-2 Marks)
    \begin{enumerate}
        \item $a^2+b^2=p^2+q^2$
        \item $\frac{1}{a^2}+\frac{1}{b^2}=\frac{1}{p^2}+\frac{1}{q^2}$
        \item $a^2+p^2=b^2+q^2$
        \item $\frac{1}{a^2}+\frac{1}{p^2}=\frac{1}{b^2}+\frac{1}{q^2}$
    \end{enumerate}
\end{enumerate}

\end{document}
