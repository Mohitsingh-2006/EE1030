%iffalse
\let\negmedspace\undefined
\let\negthickspace\undefined
\documentclass[journal,12pt,onecolumn]{IEEEtran}
\usepackage{cite}
\usepackage{amsmath,amssymb,amsfonts,amsthm}
\usepackage{algorithmic}
\usepackage{graphicx}
\usepackage{textcomp}
\usepackage{xcolor}
\usepackage{txfonts}
\usepackage{listings}
\usepackage{enumitem}
\usepackage{mathtools}
\usepackage{gensymb}
\usepackage{comment}
\usepackage[breaklinks=true]{hyperref}
\usepackage{tkz-euclide} 
\usepackage{gvv}                                        
%\def\inputGnumericTable{}                                 
\usepackage[latin1]{inputenc}     
\usepackage{xparse}
\usepackage{color}                                            
\usepackage{array}                                            
\usepackage{longtable}                                       
\usepackage{calc}                                             
\usepackage{multirow}
\usepackage{multicol}
\usepackage{hhline}                                           
\usepackage{ifthen}                                           
\usepackage{lscape}
\usepackage{tabularx}
\usepackage{array}
\usepackage{float}
\newtheorem{theorem}{Theorem}[section]
\newtheorem{problem}{Problem}
\newtheorem{proposition}{Proposition}[section]
\newtheorem{lemma}{Lemma}[section]
\newtheorem{corollary}[theorem]{Corollary}
\newtheorem{example}{Example}[section]
\newtheorem{definition}[problem]{Definition}
\newcommand{\BEQA}{\begin{eqnarray}}
\newcommand{\EEQA}{\end{eqnarray}}
\newcommand{\define}{\stackrel{\triangle}{=}}
\theoremstyle{remark}
\newtheorem{rem}{Remark}
% Marks the beginning of the document
\begin{document}
\title{gate 1}
\author{EE24Btech11041 - Mohit}
\maketitle
\renewcommand{\thefigure}{\theenumi}
\renewcommand{\thetable}{\theenumi}
\textbf{Q.7-Q.24 carry two marks each}

\begin{enumerate}
    \item The minimum number of terms required in the series expansion of $e^x$ to evaluate at $x = 1$ correct up to 3 places of decimals is
    \hfill{(XE 2007)}
    \begin{multicols}{4}
    \begin{enumerate}
        \item 8
        \item 7
        \item 6
        \item 5
    \end{enumerate}
\end{multicols}
    \item The iteration scheme $x_{n+1} = \frac{1}{1 + x_n^2}$ converges to a real number $x$ in the interval $(0, 1)$ with $x_0 = 0.5$. The value of $x$ correct up to 2 places of decimal is equal to
\hfill{(XE 2007)}
 \begin{multicols}{4}
    \begin{enumerate}
        \item 0.65
        \item 0.68
        \item 0.73
        \item 0.80
    \end{enumerate}
    \end{multicols}

    \item If the diagonal elements of a lower triangular square matrix $A$ are all different from zero, then the matrix $A$ will always be
    \hfill{(XE 2007)}
    \begin{multicols}{4}
    \begin{enumerate}
        \item symmetric
        \item non-symmetric
        \item singular
        \item non-singular
    \end{enumerate}
\end{multicols}
    \item If two eigenvalues of the matrix\\ $M = \myvec{ 2 & 6 & 0 \\ 1 & p & 0 \\ 0 & 0 & 3}$\\ are $-1$ and $4$, then the value of $p$ is:
  \hfill{(XE 2007)}
    \begin{multicols}{4}
    \begin{enumerate}
        \item 4
        \item 2
        \item 1
        \item -1
    \end{enumerate}
\end{multicols}
    \item Consider the system of linear simultaneous equations:
    \[
    x + 10y = 5;\quad y + 5z = 1; \quad 10x - y + z = 0
    \]
    On applying Gauss-Seidel method, the value of $x$ correct up to 4 decimal places is:
  \hfill{(XE 2007)}
    \begin{multicols}{4}
    \begin{enumerate}
        \item 0.0385
        \item 0.0395
        \item 0.0405
        \item 0.0410
    \end{enumerate}
\end{multicols}
    \item The graph of a function $y = f(x)$ passes through the points $(0, -3), (1, -1), (2, 3)$. Using Lagrange interpolation, the value of $x$ at which the curve crosses the $x$-axis is obtained as:
  \hfill{(XE 2007)}
    \begin{multicols}{4}
    \begin{enumerate}
    \item 1.375
    \item 0.0395
    \item 0.0405
    \item 0.0410
    \end{enumerate}
 \end{multicols}
 
    \item The equation of the straight line of best fit using the following data:
    \begin{table}[h!]    
    \centering
    \begin{tabular}[12pt]{ |c| c| c| c| c| c|}
    \hline
    x & 1 & 2 & 3 & 4 & 5 \\ 
    \hline
    y & 14 & 13 & 9 & 5 & 2 \\
    \hline   
    \end{tabular}

    \end{table}
    by the principle of least squares is:
 \hfill{(XE 2007)}
 \begin{multicols}{4}
    \begin{enumerate}
        \item $y = 18 - 3x$
        \item $y = 18.1 - 3.1x$
        \item $y = 18.2 - 3.2x$
        \item $y = 18.3 - 3.3x$
    \end{enumerate}
    \end{multicols}
    \item On solving the initial value problem:
    \begin{align}
    \frac{dy}{dx} = xy^2, \quad y(1) = 1
    \end{align}    
    by Euler's method, the value of $y$ at $x = 1.2$ with $h = 0.1$ is:
  
 \hfill{(XE 2007)}
    \begin{multicols}{4}
    \begin{enumerate}
        \item 1.1000
        \item 1.1232
        \item 1.2210
        \item 1.2331    
    \end{enumerate}
   \end{multicols}
    \item The local error of the following scheme:
   \begin{align}
    y_{n+1} = y_n + \frac{h}{12} \left( 5y_{n+1}' + 8y_n' - y_{n-1}' \right)
    \end{align}   
    by comparing with the Taylor series:
    \begin{align}
    y_{n+1} = y_n + hy_n' + \frac{h^2}{2!} y_n'' + \cdots
    \end{align}
    is:
 \hfill{(XE 2007)}
    \begin{multicols}{4}
    \begin{enumerate}
        \item $O(h^4)$
        \item $O(h^5)$
        \item $O(h^2)$
        \item $O(h^3)$
    \end{enumerate}
    \end{multicols}

    \item The area bounded by the curve $y = 1 - x^2$ and the $x$-axis from $x = -1$ to $x = 1$ using Trapezoidal rule with step length $h = 0.5$ is:
 \hfill{(XE 2007)}
    \begin{multicols}{4}
    \begin{enumerate}
        \item 1.20
        \item 1.23
        \item 1.25
        \item 1.33
    \end{enumerate}
\end{multicols}
    \item The iteration scheme:
    \begin{align}
    x_{n+1} = \sqrt{a} \left( 1 + \frac{3a^2}{x_n^2} \right) - \frac{3a^2}{x_n} , a > 0
    \end{align}
    converges to the real number:
 \hfill{(XE 2007)}
    \begin{multicols}{4}
    \begin{enumerate}
        \item $\sqrt{a}$
        \item $a$
        \item $a\sqrt{a}$
        \item $a^2$
    \end{enumerate}
\end{multicols}
    \item If the binary representation of two numbers $m$ and $n$ are $01001101$ and $00101011$, respectively, then the binary representation of $m - n$ is:
 \hfill{(XE 2007)}
    \begin{multicols}{4}
    \begin{enumerate}
        \item 00010010
        \item 00100010
        \item 00111101
        \item 00100001
    \end{enumerate}
\end{multicols}
    \item Which of the following statements are true in a C program?
    
         P: A local variable is used only within the block where it is defined, and its sub-blocks\\
         Q: Global variables are declared outside the scope of all blocks\\
         R: Extern variables are used by linkers for sharing between other compilation units\\
         S: By default, all global variables are extern variables
 \hfill{(XE 2007)}
    \begin{multicols}{4}
    \begin{enumerate}
        \item P and Q
        \item P, Q and R
        \item P, Q and S
        \item P, Q, R and S
    \end{enumerate}

    \end{multicols}{4}


\lstset{
  basicstyle=\ttfamily,
  keywordstyle=\color{black},
  commentstyle=\color{black},
  stringstyle=\color{black},
  showstringspaces=false,
  numbers=left,
  numberstyle=\tiny\color{gray},
  frame=single,
  breaklines=true,
  captionpos=b,
  tabsize=4
}


\item Consider the following recursive function $g()$.\\

\lstinputlisting{codes/code1}


Which value will be returned if the function g  is called with 6, 6 ?
\hfill{(XE 2007)}
\begin{multicols}{4}
\begin{enumerate}
\item 2
\item 4
\item 6
\item 8
\end{enumerate}
 \end{multicols}

\item If the following function is called with  x = 1 \\

\lstinputlisting{codes/code2}



The value returned will be close to
\hfill{(XE 2007)}
\begin{multicols}{4}
\begin{enumerate}
\item $\log_e 2$ 
\item $\log_e 3 $
\item $1 + e$ 
\item $e $
\end{enumerate}
\end{multicols}
    
\item Consider the following C program \\ 

\lstinputlisting[language=C]{codes/code3.c}

Which number will be printed if the input string is 10110?
\hfill{(XE 2007)}
\begin{multicols}{4}
\begin{enumerate}
\item 31
\item 28
\item 25
\item 22
\end{enumerate}    
\end{multicols}


Consider the following C program segment\\ 

\item The value of sum that will be printed by the program is
\hfill{(XE 2007)}
\lstinputlisting[language=C]{codes/code4.c}
\begin{multicols}{4}
\begin{enumerate}
\item 369
\item 361
\item 303
\item 261
\end{enumerate} 
\end{multicols}

\end{enumerate}
\end{document}
