%iffalse
\let\negmedspace\undefined
\let\negthickspace\undefined
\documentclass[journal,12pt,onecolumn]{IEEEtran}
\usepackage{cite}
\usepackage{amsmath,amssymb,amsfonts,amsthm}
\usepackage{algorithmic}
\usepackage{graphicx}
\usepackage{textcomp}
\usepackage{xcolor}
\usepackage{txfonts}
\usepackage{listings}
\usepackage{enumitem}
\usepackage{mathtools}
\usepackage{gensymb}
\usepackage{comment}
\usepackage[breaklinks=true]{hyperref}
\usepackage{tkz-euclide} 
\usepackage{gvv}                                        
%\def\inputGnumericTable{}                                 
\usepackage[latin1]{inputenc}     
\usepackage{xparse}
\usepackage{color}                                            
\usepackage{array}                                            
\usepackage{longtable}                                       
\usepackage{calc}                                             
\usepackage{multirow}
\usepackage{multicol}
\usepackage{hhline}                                           
\usepackage{ifthen}                                           
\usepackage{lscape}
\usepackage{tabularx}
\usepackage{array}
\usepackage{float}
\usepackage{circuitikz}
\usepackage{tikz}
\newtheorem{theorem}{Theorem}[section]
\newtheorem{problem}{Problem}
\newtheorem{proposition}{Proposition}[section]
\newtheorem{lemma}{Lemma}[section]
\newtheorem{corollary}[theorem]{Corollary}
\newtheorem{example}{Example}[section]
\newtheorem{definition}[problem]{Definition}
\newcommand{\BEQA}{\begin{eqnarray}}
\newcommand{\EEQA}{\end{eqnarray}}
\newcommand{\define}{\stackrel{\triangle}{=}}
\theoremstyle{remark}
\newtheorem{rem}{Remark}
% Marks the beginning of the document
\begin{document}
\title{gate 2}
\author{EE24Btech11041 - Mohit}
\maketitle
\renewcommand{\thefigure}{\theenumi}
\renewcommand{\thetable}{\theenumi}



\textbf{Q.1 - Q.25 carry one mark each}
\begin{enumerate}
\item Consider an anti-symmetric tensor $P_{ij}$ with the indices $i$ and $j$ running from 1 to 5. The number of independent components of the tensor is 
\hfill{(PH 2010)}
\begin{multicols}{4}
\begin{enumerate}
\item 3
\item 10
\item 9 
\item 6
\end{enumerate}
\end{multicols}
\item The value of the integral $\int_{c} \frac {e^z \sin{z}}{z^2}dz$, where the contour $C$ is the unit circle:
$\abs{z-2}=1$, is

\hfill{(PH 2010)}
\begin{multicols}{4}
\begin{enumerate}
\item $2\pi i$
\item $4\pi i$
\item $\pi i$
\item 0
\end{enumerate}
\end{multicols}

\item The eigenvalues of the matrix \myvec{
2 & 3 & 0 \\
3 & 2 & 0 \\
0 & 0 & 1 \\
}
are 
\hfill{(PH 2010)}
\begin{multicols}{4}
\begin{enumerate}
\item 5,2,-2
\item -5,-1,-1
\item 5,1,-1
\item -5,1,1
\end{enumerate}
\end{multicols}
\item If 
$f(x) =$ 
$\begin{cases} 
0 \quad for \quad x < 3, \\ 
x - 3 \quad for \quad x \geq 3, 
\end{cases}$

then the Laplace transform of $f(x)$ is 

\hfill{(PH 2010)}
\begin{multicols}{4}
\begin{enumerate}
\item $s^{-2}e^{3x}$
\item $s^{2}e^{-3x}$
\item $s^{-2}$ 
\item $s^{-2}e^{-3x}$
\end{enumerate}
\end{multicols}

\item The valence electrons do not directly determine the following property of a metal. 

\hfill{(PH 2010)}
\begin{enumerate}
\item Electrical conductivity
\item Thermal conductivity
\item Shear modulus
\item Metallic lustre
\end{enumerate}


\item Consider X-ray diffraction from a crystal with a face-centered-cubic (fcc) lattice. The lattice plane for which there is NO diffraction peak is 

\hfill{(PH 2010)}
\begin{multicols}{4}
\begin{enumerate}
\item (2, 1, 2)
\item (1, 1, 1)
\item (2, 0, 0)
\item (3, 1, 1)
\end{enumerate}
\end{multicols}


\item The Hall coefficient, $R_H$, of sodium depends on 

\hfill{(PH 2010)}

\begin{enumerate}
\item The effective charge carrier mass and carrier density 
\item The charge carrier density and relaxation time
\item The charge carrier density only 
\item The effective charge carrier mass
\end{enumerate}


\item The Bloch theorem states that within a crystal, the wavefunction, $\psi(\vec{r})$, of an electron has the form 

\hfill{(PH 2010)}

\begin{enumerate}
\item $\psi(\vec{r}) = u(\vec{r}) e^{i \vec{k} \cdot \vec{r}}$ where $u(\vec{r})$ is an arbitrary function and $\vec{k}$ is an arbitrary vector 
\item $\psi(\vec{r}) = u(\vec{r}) e^{i \vec{G} \cdot \vec{r}}$ where $u(\vec{r})$ is an arbitrary function and $\vec{G}$ is a reciprocal lattice vector 
\item $\psi(\vec{r}) = u(\vec{r} + \vec{\Lambda}) e^{i \vec{G} \cdot \vec{r}}$ where $u(\vec{r}) = u(\vec{r} + \vec{\Lambda})$, $\vec{\Lambda}$ is a lattice vector and $\vec{G}$ is a reciprocal lattice vector 
\item $\psi(\vec{r}) = u(\vec{r} + \vec{\Lambda}) e^{i \vec{k} \cdot \vec{r}}$ where $u(\vec{r}) = u(\vec{r} + \vec{\Lambda})$, $\vec{\Lambda}$ is a lattice vector and $\vec{k}$ is an arbitrary vector
\end{enumerate}


\item In an experiment involving a ferromagnetic medium, the following observations were made. Which one of the plots does NOT correctly represent the property of the medium? ($T_c$ is the Curie temperature) 

\hfill{(PH 2010)}
\begin{multicols}{2}
\begin{enumerate}

\item {\scalebox{0.4}{\begin{tikzpicture}
\tikzstyle{every node}=[font=\small]
\draw [short] (7.75,7.75) .. controls (11,15) and (11.25,14) .. (18.25,14.25);
\draw [->, >=Stealth] (5.5,7.75) -- (18.75,7.75);
\draw [->, >=Stealth] (5.5,7.75) -- (5.5,16.5);
\node [font=\Large, rotate around={90:(0,0)}] at (4.25,12.25) {Spontantous Magnetization};
\node [font=\Large] at (18.5,7) {1/T};
\node [font=\Large] at (7.75,7) {1/$T_c$};
\draw [dashed] (7.75,7.75) -- (7.75,16.25);
\end{tikzpicture}}}


\item 
  {\scalebox{0.7}{ \begin{tikzpicture}
    \draw[fill=white!70] (-3,-3) .. controls (2.5,-3) and (-0.5,3) .. (3,3)
             .. controls (-2.5,3) and (0.5,-3) ..(-3,-3);
    \draw[-latex] (-4,0) -- (4,0)node[below]{Magnetic field};
    \draw[-latex] (0,-4) -- (0,4)node[left]{Magnetization};
            
    \draw[dashed] (-4,3) -- (4,3);
    \draw[dashed] (-4,-3) -- (4,-3);

\end{tikzpicture}}
}


\item {\scalebox{0.4}{ 
\begin{tikzpicture}
\tikzstyle{every node}=[font=\LARGE]
\draw [short] (8,6.5) .. controls (16,7.75) and (17.75,12.75) .. (20,17);
\draw [->, >=Stealth] (8,6.5) -- (8,17);
\draw [->, >=Stealth] (8,6.5) -- (21.5,6.5);
\node [font=\LARGE] at (21.25,5.75) {T};
\node [font=\LARGE, rotate around={90:(0,0)}] at (6.75,11.25) {Specific Heat};
\node [font=\LARGE] at (13.75,6) {$T_c$};
\draw [dashed] (13.5,6.5) -- (13.5,16.75);
\end{tikzpicture}
}
} 



\item {
\scalebox{0.5}{
\begin{tikzpicture}
\tikzstyle{every node}=[font=\LARGE]
\draw [short] (11,14.25) .. controls (11.25,13.25) and (11.5,9.75) .. (21,8.5);
\draw [->, >=Stealth] (6.75,8) -- (22.5,8);
\draw [dashed] (10.75,14.25) -- (10.75,8);
\draw [short] (10.5,14.25) .. controls (10.5,13.25) and (10,9.5) .. (5.75,8.5);
\draw [->, >=Stealth] (5.75,8) -- (5.75,16);
\draw [->, >=Stealth] (5.75,8) -- (22.5,8);
\node [font=\LARGE] at (10.75,7.5) {$T_c$};
\node [font=\LARGE] at (22.25,7.5) {T};
\node [font=\LARGE, rotate around={90:(0,0)}] at (5,12.25) {Magnetic Susceptibility};
\end{tikzpicture}
}
}

\end{enumerate}
\end{multicols}
\item The thermal conductivity of a given material reduces when it undergoes a transition from its normal state to the superconducting state. The reason is: 

\hfill{(PH 2010)}

\begin{enumerate}
\item The Cooper pairs cannot transfer energy to the lattice 
\item Upon the formation of Cooper pairs, the lattice becomes less efficient in heat transfer
\item The electrons in the normal state lose their ability to transfer heat because of their coupling to the Cooper pairs 
\item  The heat capacity increases on transition to the superconducting state leading to a reduction in thermal conductivity
\end{enumerate}



\item The basic process underlying the neutron $\beta$- decay is 


\hfill{(PH 2010)}
\begin{multicols}{2}
\begin{enumerate}
\item $d \rightarrow u + e^- + \overline{\nu}_e$
\item $d \rightarrow u + e^-$ 
\item $s \rightarrow u + e^- + \overline{\nu}_e$ 
\item $u \rightarrow d + e^- + \overline{\nu}_e$
\end{enumerate}
\end{multicols}

\item In the nuclear shell model, the spin parity of $^{15}N$ is given by 

\hfill{(PH 2010)}
\begin{multicols}{4}
\begin{enumerate}
\item $\frac{1}{2}^{-}$ 
\item $\frac{1}{2}^{+}$ 
\item $\frac{3}{2}^{-}$ 
\item $\frac{3}{2}^{+}$ 
\end{enumerate}
\end{multicols}



\item Match the reactions on the left with the associated interactions on the right. \\
(1) $\pi^+ \rightarrow \mu^+ + \nu_{\mu}$  \quad \qquad \qquad \qquad (i) Strong \\
(2) $\pi^0 \rightarrow \gamma + \gamma$ \qquad \qquad \qquad \qquad (ii) Electromagnetic\\ 
(3) $\pi^0 + n \rightarrow \pi^- + p$ \qquad \qquad \qquad (iii) Weak 


\hfill{(PH 2010)}
\begin{multicols}{2}
\begin{enumerate}
\item (1, iii), (2, ii), (3, i) 
\item (1, i), (2, ii), (3, iii) 
\item (1, ii), (2, i), (3, iii) 
\item (1, iii), (2, i), (3, ii)
\end{enumerate}
\end{multicols}



\end{enumerate}
\end{document}
