%iffalse
\let\negmedspace\undefined
\let\negthickspace\undefined
\documentclass[journal,12pt,onecolumn]{IEEEtran}
\usepackage{cite}
\usepackage{amsmath,amssymb,amsfonts,amsthm}
\usepackage{algorithmic}
\usepackage{graphicx}
\usepackage{textcomp}
\usepackage{xcolor}
\usepackage{txfonts}
\usepackage{listings}
\usepackage{enumitem}
\usepackage{mathtools}
\usepackage{gensymb}
\usepackage{comment}
\usepackage[breaklinks=true]{hyperref}
\usepackage{tkz-euclide} 
\usepackage{gvv}                                        
%\def\inputGnumericTable{}                                 
\usepackage[latin1]{inputenc}     
\usepackage{xparse}
\usepackage{color}                                            
\usepackage{array}                                            
\usepackage{longtable}                                       
\usepackage{calc}                                             
\usepackage{multirow}
\usepackage{multicol}
\usepackage{hhline}                                           
\usepackage{ifthen}                                           
\usepackage{lscape}
\usepackage{tabularx}
\usepackage{array}
\usepackage{float}
\usepackage{circuitikz}
\usepackage{tikz}
\usepackage{pgfplots}
\pgfplotsset{compat=1.18}
\usetikzlibrary{arrows.meta}
\usetikzlibrary{decorations.pathmorphing}
\newtheorem{theorem}{Theorem}[section]
\newtheorem{problem}{Problem}
\newtheorem{proposition}{Proposition}[section]
\newtheorem{lemma}{Lemma}[section]
\newtheorem{corollary}[theorem]{Corollary}
\newtheorem{example}{Example}[section]
\newtheorem{definition}[problem]{Definition}
\newcommand{\BEQA}{\begin{eqnarray}}
\newcommand{\EEQA}{\end{eqnarray}}
\newcommand{\define}{\stackrel{\triangle}{=}}
\theoremstyle{remark}
\newtheorem{rem}{Remark}
% Marks the beginning of the document
\begin{document}
\title{gate 4}
\author{EE24Btech11041 - Mohit}
\maketitle
\renewcommand{\thefigure}{\theenumi}
\renewcommand{\thetable}{\theenumi}


\begin{enumerate}
\item Rajiv Gandhi Khel Ratna Award was conferred \rule{2cm}{0.4pt} Mary Kom , a six-time world champion in boxing, recently in a ceremony \rule{2cm}{0.4pt} the Rashtrapati Bhawan (the President's official residence) in New Delhi .
\hfill{(MA 2020)}

\begin{enumerate}
\item with, at
\item on, in
\item on, at
\item to, at
\end{enumerate}

\item Despite a string of poor performances , the chances of K.L.Rahul's selection in the team are \rule{2cm}{0.4pt} .
\hfill{(MA 2020)}
\begin{enumerate}
\item slim
\item bright
\item obvious
\item uncertain
\end{enumerate}

\item Select the world that fits the analogy :\\
\\
Cover : Uncover :: Associate :  \rule{2cm}{0.4pt}
\hfill{(MA 2020)}
\begin{enumerate}
\item Unassiociate
\item Inassiociate
\item Misassociate
\item Dissociate
\end{enumerate}

\item  	Hit by floods, the kharif (summer sown) crops in various parts of the country have been affected. Officials believe that the loss in production of the kharif crops can be recovered in the output of the rabi (winter sown) crops so that the country can achieve its food-grain production target of 291 million tons in the crop year 2019-20 (July-June) . They are hopeful that good rains in July-August will help the soil retain moisture for a longer period , helping winter sown crops such as wheat and pulses during the November-February period .
\\
\\
Which of the following statments can be inferred from the given passage
\hfill{(MA 2020)}
\begin{enumerate}
\item Officials declared that the food-grain production target will be met due to good grains .
\item Officials declared that the food-grain production target to be met by the November-February period .
\item Officials declared that the food-grain production target cannot be met due floods .
\item Officials declared that the food-grain production target will be met due to good rabi produce .
\end{enumerate}
 
 
 \item The difference between the sum of the first $2n$ natural numbers and the sum of the first $n$ odd natural numbers is \rule{2cm}{0.4pt} .
  \hfill{(MA 2020)}
 \begin{enumerate}
\item $n^2-n$
\item $n^2n$
\item $2n^2-n$
\item $2n^2+n$
\end{enumerate}


 \item Repo rate is the rate at which Reserve Bank of India (RBI) lends commercial banks , and reverse repo rate is the rate at which RBI borrows money from commercial banks .
 \hfill{(MA 2020)}
 \\ 
Which of the following statement can be inferred from the above passage ?

 \begin{enumerate}
\item Decrease in repo rate will increase cost of borrowing and decrease lending by commercial banks .
\item Increase in repo rate will decrease cost of borrowing and increase lending by commercial banks .
\item Increase in repo rate will decrease cost of borrowing and decrease lending by commercial banks .
\item Decrease in repo rate will decrease cost of borrowing and increase lending by commercial banks .
\end{enumerate}

\item P,Q,R,S,T,U,V and W are seated around a circular table.
\hfill{(MA 2020)}
\begin{enumerate}
\item S is seated opposite to W .
\item U is seated in the second place to the right of R .
\item T is seated at the third place to the left of R .
\item V is neighbour of S .
\end{enumerate}

\begin{enumerate}
\item P is neighbour of R
\item Q is neighbour of R
\item P is not seated at the third place to the left of R
\item R is the left neighbour of S 
\end{enumerate}

\item The distance between Delhi and Agra is 233 km . A car $P$ started travelling from Delhi to Agra and another car $Q$ started from Agra to Delhi along the same road 1 hour after the car $P$ started . The two cars crossed each other 75 minutes after the car $Q$ started. Both cars were travelling at constant speed . The speed of car $P$ was $10 km/hr$ more than the speed of car $Q$ . How many kilometers the car $Q$ had travelled when the cars crossed each other ? 
\hfill{(MA 2020)}
\begin{enumerate}
\item 66.6
\item 75.2
\item 88.2
\item 116.5
\end{enumerate}
\item For a matrix $M$ = $[m_{ij}]$ ; i, j = 1, 2, 3, 4, the diagonal elemenets are all zero ans $m_{ij}$ = $-m_{ji}$.\\The minimum number of elements required to fully specify the matrix is \rule{2cm}{0.4pt}.
\hfill{(MA 2020)}
\begin{enumerate}
\item 0
\item 6
\item 12
\item 16
\end{enumerate}

\item The profit shares of two companies P and Q are shown in the figure . If the two companies have invested a fixed and equal amount every year , then the ratio of the total revenue of company P to the total revenue of company Q , during 2013 - 2018 \rule{2cm}{0.4pt}.
\hfill{(MA 2020)}

\begin{center}
\begin{circuitikz}
\tikzstyle{every node}=[font=\small]
\draw [line width=1pt, short] (6.25,6) -- (16.25,6);
\draw [line width=1pt, short] (6.25,6) -- (6.25,14.75);
\draw [line width=1pt, short] (6.25,14.75) -- (16.25,14.75);
\draw [line width=1pt, short] (16.25,14.75) -- (16.25,6);
\draw [line width=0.2pt, short] (6.25,7.25) -- (16.25,7.25);
\draw [line width=0.2pt, short] (16.25,8.5) -- (6.25,8.5);
\draw [line width=0.2pt, short] (6.25,9.75) -- (16.25,9.75);
\draw [line width=0.2pt, short] (16.25,11) -- (6.25,11);
\draw [line width=0.2pt, short] (6.25,12.25) -- (16.25,12.25);
\draw [line width=0.2pt, short] (6.25,13.5) -- (16.25,13.5);
\draw [line width=0.2pt, short] (7.5,14.75) -- (7.5,6);
\draw [line width=0.2pt, short] (8.75,14.75) -- (8.75,6);
\draw [line width=0.2pt, short] (10,14.75) -- (10,6);
\draw [line width=0.2pt, short] (11.25,14.75) -- (11.25,6);
\draw [line width=0.2pt, short] (12.5,14.75) -- (12.5,6);
\draw [line width=0.2pt, short] (13.75,14.75) -- (13.75,6);
\node [font=\normalsize] at (8,5.5) {2013};
\node [font=\normalsize] at (9.25,5.5) {2014};
\node [font=\normalsize] at (10.75,5.5) {2015};
\node [font=\normalsize] at (11.75,5.5) {2016};
\node [font=\normalsize] at (13.25,5.5) {2017};
\node [font=\normalsize] at (14.25,5.5) {2018};
\draw [line width=0.2pt, short] (15,14.75) -- (15,6);
\node [font=\normalsize] at (6,6) {0};
\node [font=\normalsize] at (5.75,7.25) {10};
\node [font=\normalsize] at (5.75,8.5) {20};
\node [font=\normalsize] at (5.75,9.75) {30};
\node [font=\normalsize] at (5.75,11) {40};
\node [font=\normalsize] at (5.75,12.25) {50};
\node [font=\normalsize] at (5.75,13.5) {60};
\node [font=\normalsize] at (5.75,14.75) {70};
\node [font=\large] at (10.5,4.75) {year};
\node [font=\large, rotate around={90:(0,0)}] at (4.25,10.5) {Profit percentage};
\draw [ fill={rgb,255:red,224; green,27; blue,36} , line width=0.2pt , rotate around={90:(7, 14.375)}] (7,14.25) rectangle (7,14.5);
\draw [ fill={rgb,255:red,224; green,27; blue,36} , line width=0.2pt , rotate around={90:(7, 14.25)}] (6.75,14.5) rectangle (7.25,14);
\draw [ fill={rgb,255:red,53; green,132; blue,228} , line width=0.2pt , rotate around={90:(7, 13.5)}] (7,13.75) rectangle (7,13.25);
\draw [ fill={rgb,255:red,53; green,132; blue,228} , line width=0.2pt , rotate around={90:(7, 13.5)}] (6.75,13.75) rectangle (7.25,13.25);
\node [font=\small] at (8.1,14.25) {Company P};
\node [font=\small] at (8.1,13.5) {Company Q};
\draw [ fill={rgb,255:red,224; green,27; blue,36} , line width=0.2pt ] (7.75,7.25) rectangle (8,6);
\draw [ fill={rgb,255:red,224; green,27; blue,36} , line width=0.2pt ] (9,8.5) rectangle (9.25,6);
\draw [ fill={rgb,255:red,224; green,27; blue,36} , line width=0.2pt ] (10.25,11) rectangle (10.5,6);
\draw [ fill={rgb,255:red,224; green,27; blue,36} , line width=0.2pt ] (11.5,11) rectangle (11.75,6);
\draw [ fill={rgb,255:red,224; green,27; blue,36} , line width=0.2pt ] (12.75,12.25) rectangle (13,6);
\draw [ fill={rgb,255:red,224; green,27; blue,36} , line width=0.2pt ] (14,11) rectangle (14.25,6);
\draw [ fill={rgb,255:red,53; green,132; blue,228} , line width=0.2pt ] (8,8.5) rectangle (8.25,6);
\draw [ fill={rgb,255:red,53; green,132; blue,228} , line width=0.2pt ] (9.25,9.75) rectangle (9.5,6);
\draw [ fill={rgb,255:red,53; green,132; blue,228} , line width=0.2pt ] (10.5,9.75) rectangle (10.75,6);
\draw [ fill={rgb,255:red,53; green,132; blue,228} , line width=0.2pt ] (11.75,12.25) rectangle (12,6);
\draw [ fill={rgb,255:red,53; green,132; blue,228} , line width=0.2pt ] (13,13.5) rectangle (13.25,6);
\draw [ fill={rgb,255:red,53; green,132; blue,228} , line width=0.2pt ] (14.25,13.5) rectangle (14.5,6);
\end{circuitikz}
\end{center}

\begin{enumerate}
\item 15:17
\item 16:17
\item 17:15
\item 17:16
\end{enumerate}



\item Suppose that $T_1$ and $T_2$ are topologies on $X$ include by metrics $d_1$ and $d_2$,respectively, such that $T_1 \subseteq T_2$.Then which of the following statments is TRUE ?
\hfill{(MA 2020)}
\begin{enumerate}
\item If a sequence sonverges in $\brak{X,d_2}$ then it converges in $\brak{X,d_1}$
\item If a sequence sonverges in $\brak{X,d_1}$ then it converges in $\brak{X,d_2}$
\item Every open ball in $\brak{X,d_1}$ is an open ball in $\brak{X,d_2}$
\item The map $x \rightarrow x$ from $\brak{X,d_1}$ to $\brak{X,d_2}$ is continuous
\end{enumerate}

\item Let $D = [-1,1]\times [-1,1]$. if the function $f:D \rightarrow \mathbb{R}$ is defined by \\

$f(x)=$
$ \begin{cases}
   \frac{x^2-y^2}{\brak{x^2+y^2}^2},\quad \brak{x,y} \neq \brak{0,0}\\
   0, \quad \brak{x,y} = \brak{0,0}
\end{cases} $ \\
then
\hfill{(MA 2020)}

\begin{enumerate}
\item $f$ is continuous at $\brak{0,0}$
\item both the first order partial derivatives of $f$ exist at $\brak{0,0}$ 
\item $\int \int_D \abs{f(x,y)}^{\frac{1}{2}}dx \text{ } dy$ is finite
\item $\int \int_D \abs{f(x,y)}\text{ }dx \text{ } dy$ is finite
\end{enumerate}


\item The initial value problem
\hfill{(MA 2020)}
\begin{align}
y^{'}=y^{\frac{3}{5}}, \quad y\brak{0}=b
\end{align}
has

\begin{enumerate}
\item a unique solution if $b=0$
\item no solution if $b=1$
\item infinitely many solutions if $b=2$
\item a unique solution if $b=1$
\end{enumerate}


\end{enumerate}
\end{document}
