%iffalse
\let\negmedspace\undefined
\let\negthickspace\undefined
\documentclass[journal,12pt,onecolumn]{IEEEtran}
\usepackage{cite}
\usepackage{amsmath,amssymb,amsfonts,amsthm}
\usepackage{algorithmic}
\usepackage{graphicx}
\usepackage{textcomp}
\usepackage{xcolor}
\usepackage{txfonts}
\usepackage{listings}
\usepackage{enumitem}
\usepackage{mathtools}
\usepackage{gensymb}
\usepackage{comment}
\usepackage[breaklinks=true]{hyperref}
\usepackage{tkz-euclide} 
\usepackage{listings}
\usepackage{gvv}                                        
%\def\inputGnumericTable{}                                 
\usepackage[latin1]{inputenc}     
\usepackage{xparse}
\usepackage{color}                                            
\usepackage{array}                                            
\usepackage{longtable}                                       
\usepackage{calc}                                             
\usepackage{multirow}
\usepackage{multicol}
\usepackage{hhline}                                           
\usepackage{ifthen}                                           
\usepackage{lscape}
\usepackage{tabularx}
\usepackage{array}
\usepackage{float}
\newtheorem{theorem}{Theorem}[section]
\newtheorem{problem}{Problem}
\newtheorem{proposition}{Proposition}[section]
\newtheorem{lemma}{Lemma}[section]
\newtheorem{corollary}[theorem]{Corollary}
\newtheorem{example}{Example}[section]
\newtheorem{definition}[problem]{Definition}
\newcommand{\BEQA}{\begin{eqnarray}}
\newcommand{\EEQA}{\end{eqnarray}}
\newcommand{\define}{\stackrel{\triangle}{=}}
\theoremstyle{remark}
\newtheorem{rem}{Remark}
% Marks the beginning of the document
\begin{document}
\title{JEE PYQ 1}
\author{EE24Btech11041 - Mohit}
\maketitle
\renewcommand{\thefigure}{\theenumi}
\renewcommand{\thetable}{\theenumi}
\begin{enumerate}
\item Let $f(x)=3\sqrt{x-2}+\sqrt{4-x}$ be a real function . If $\alpha$ and $\beta$ are respectively the minimum and the maximum values of $f$ , then $\alpha^2$+2$\beta^2$ is equal to
\hfill{(April-2024)}
\begin{multicols}{2}
\begin{enumerate}
\item 44
\item 42 
\item 24
\item 31
\end{enumerate}
\end{multicols}
\item Let $A=\myvec{1&2\\0&1}$ and $B=I+\operatorname{adj}\brak{A}+\brak{\operatorname{adj}\brak{A}}^2+\dots+\brak{\operatorname{adj}\brak{A}}^{10}$.\\
Then,the sum of all the elements of the matrix $B$ is:
\hfill{(April-2024)}
\begin{multicols}{2}
\begin{enumerate}
\item -110
\item -88 
\item 22
\item -124
\end{enumerate}
\end{multicols}
\item Let three real numbers $a,b,c$ be in arithmetic progression $a+1,b,c+3$ be in geometric progession . If $a>10$ and the arithmetic mean of $a,b$ and $c$ is 8 , then the cube of the geometric mean of $a,b$ and $c$ is
\hfill{(April-2024)}
\begin{multicols}{2}
\begin{enumerate}
\item 316
\item 120
\item 128
\item 312
\end{enumerate}
\end{multicols}
\item Let a relation R on $\mathbb{N} \times \mathbb{N}$ be defined as:\\
$\brak{x_1,y_1}$ R $\brak{x_2,y_2}$ if and only if $x_1\leq x_2$ or $y_1\leq y_2$.\\
Consider the two statments:
\begin{enumerate}
\item R is reflexive but not symmetric.
\item R is transitive
\end{enumerate}
Then which one of the following is true?\\
\hfill{(April-2024)}
\begin{multicols}{2}
\begin{enumerate}
\item Neither $\brak{1}$ nor $\brak{2}$ is correct. 
\item Only $\brak{2}$ is correct.
\item Only $\brak{1}$ is correct. 
\item Both $\brak{1}$ and $\brak{2}$ are correct.
\end{enumerate}
\end{multicols}
\item Given that the inverse trigonometric function assumes principal values only . Let $x$,$y$ be any two real numbers in $[-1,1]$ such that  $\cos^{-1}(x)-\sin^{-1}(y)$=$\alpha $,$-\frac{\pi}{2}<\alpha<\pi$.
Then , the minimum value of $x^2-y^2+2xy\sin{\alpha}$ is 
\hfill{(April-2024)}
\begin{multicols}{2}
\begin{enumerate}
\item -1
\item 0
\item $\frac{1}{2}$
\item $-\frac{1}{2}$
\end{enumerate}
\end{multicols}
\item If the function\\
\\$f(x)=$
$ \begin{cases}
   \frac{72^x-9^x-8^x+1}{\sqrt{2}-\sqrt{1+\cos{x}}},x \neq 0\\
   a\log2\log3,x=0
\end{cases} $ \\
is continious at $x=0$ , then the value of $a^2$ is equal to 
\hfill{(April-2024)}
\begin{multicols}{2}
\begin{enumerate}
\item 746
\item 968 
\item 1250
\item 1152
\end{enumerate}
\end{multicols}
\item Let $C$ be a circle with radius $\sqrt{10}$ units and centre at the origin . Let the line $x+y=2$ intersets the circle $C$ at the points $P$ and $Q$ . Let $MN$ be a chord of $C$ of length 2 unit and slope -1 . Then , the distance (in units) between the chords $PQ$ and the chords $MN$ is
\hfill{(April-2024)}
\begin{multicols}{2}
\begin{enumerate}
\item 2-$\sqrt{3}$
\item $\sqrt{2}$+1 
\item $\sqrt{2}$-1
\item 3-$\sqrt{2}$
\end{enumerate}
\end{multicols}
\item If the mean of the following probability distribution of a radian variable X:\\

\begin{table}[h!]    
  \centering
 \begin{tabular}[12pt]{ |c| c| c| c| c| c|}
    \hline
    x & 1 & 2 & 3 & 4 & 5 \\ 
    \hline
    y & 14 & 13 & 9 & 5 & 2 \\
    \hline   
    \end{tabular}

\end{table}
is $\frac{46}{9}$, then the variance of the distribution is \\
\hfill{(April-2024)}
\begin{multicols}{2}
\begin{enumerate}
\item $\frac{566}{81}$
\item  $\frac{173}{27}$
\item $\frac{581}{81}$
\item $\frac{151}{81}$
\end{enumerate}
\end{multicols}
\item The area (in sq. units) of the region \\
S=\{ $\vec{z} \in \mathbb{C}:\abs{z - 1} \leq 2$ ; $\brak{z + \bar{z}} + i\brak{z - \bar{z}} \leq 2$, $Im(z) \geq 0 \}$ is
\hfill{(April-2024)}
\begin{multicols}{2}
\begin{enumerate}
\item $\frac{7\pi}{3}$
\item $\frac{7\pi}{4}$
\item $\frac{17\pi}{8}$
\item $\frac{3\pi}{2}$
\end{enumerate}
\end{multicols}
\item Let $\vec{a} = \hat{i} + \hat{j} + \hat{k}$, $\vec{b} = 2\hat{i} + 4\hat{j} - 5\hat{k}$, and $\quad \vec{c} = x\hat{i} + 2\hat{j} + 3\hat{k}$, $\quad x \in \mathbb{R}$. \\
If $\vec{d}$  is the unit vector in the direction of  $\vec{b} + \vec{c}$ such that $ \vec{a}.\vec{d} = 1$,then ($\vec{a} \times \vec{b}$).$\vec{c}$ is equal to 
\hfill{(April-2024)}
\begin{multicols}{2}
\begin{enumerate}
\item 3
\item 6
\item 11
\item 9
\end{enumerate}
\end{multicols}

\item Let $P$ be the point of intersection of the lines $\frac{x-2}{1}=\frac{y-4}{5}=\frac{z-2}{1}$ and $\frac{x-3}{2}=\frac{y-2}{3}=\frac{z-3}{2}$ . Then , the shortest distance of $P$ from the line $4x=2y=z$ is
\hfill{(April-2024)}
\begin{enumerate}
\item $\frac{\sqrt{14}}{7}$
\item  $\frac{6\sqrt{14}}{7}$
\item $\frac{5\sqrt{14}}{7}$
\item $\frac{3\sqrt{14}}{7}$
\end{enumerate}

\item Let $y=y(x)$ be the solution of the differential equation \\
$\brak{x^2+4}^2dy$ + $\brak{2x^3y+8xy-2}dx$=0.If $y(0)-0$,Then $y(2)$ is equal to 
\hfill{(April-2024)}
\begin{multicols}{2}
\begin{enumerate}
\item $\frac{\pi}{32}$
\item $\frac{\pi}{8}$
\item $\frac{\pi}{16}$
\item $2\pi$
\end{enumerate}
\end{multicols}
\item For $\lambda > 0$, let $\theta$ be the angle between the vectors $\mathbf{a} = \hat{i} + \lambda \hat{j} - 3 \hat{k}$ and $\mathbf{b} = 3 \hat{i} - \hat{j} + 2 \hat{k}$. \\
If the vectors $\mathbf{a} + \mathbf{b}$ and $\mathbf{a} - \mathbf{b}$ are mutually perpendicular, then the value of $(14 \cos \theta)^2$ is equal to: \\
\hfill{(April-2024)}
\begin{multicols}{2}
\begin{enumerate}
\item 40
\item 25
\item 50
\item 20
\end{enumerate}
\end{multicols}
\item The area (in sq. units) of the region described by \{$(x, y)$ : $y^2 \leq 2x$,and $y \geq 4x - 1$ \} is
\hfill{(April-2024)}
\begin{multicols}{2}
\begin{enumerate}
\item $\frac{11}{32}$
\item $\frac{11}{12}$
\item $\frac{9}{32}$
\item $\frac{8}{9}$
\end{enumerate}
\end{multicols}
\item Let $PQ$ be a chord of parabola $y^2=12x$ and the midpoint of PQ be at \brak{4,1} . Then ,which of the following points lies on the line on the line passing through the points $P$ and $Q$ ?
\hfill{(April-2024)}
\begin{multicols}{2}
\begin{enumerate}
\item \brak{3,-3}
\item \brak{\frac{3}{2},-16}
\item \brak{\frac{1}{2},-20}
\item \brak{2,-9}
\end{enumerate}
\end{multicols}
\end{enumerate}
\end{document}
